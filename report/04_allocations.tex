\section{IP address allocation dynamics}
\label{sec:allocations}

\subsection{Allocated IP block sizes}

\begin{figure}[htbp]
 	\centering
 		\includegraphics[width=0.5\textwidth]{02_prefixes/02_ip_prefixes_zoom}
	\caption{Allocated prefix distribution}
 	\label{fig:IP allocations}
\end{figure}

The distribution of allocated block sizes has changed over the past six years, as shown in figure \ref{fig:IP allocations}. The number of allocations has increased for every prefix prefix length, but at different rates. In 2003, /24 was the most popular size, as it is today, but there were only about half as many /20 allocations as /19 ones, whereas in 2008 they are about the same. There has been almost no increase in /16 blocks. Throughout the siz-year period, most IP allocations are /24 blocks. The next most popular block size is /16, followed by /19 and /20. The smaller block sizes (/19 and smaller) have had a larger increase in usage than /16, /17, and /18.

\subsection{Yearly distribution of IP allocations}

% \begin{figure}[htbp]
%  	\centering
%  		\includegraphics[width=0.5\textwidth]{04_plus_minus/addremoveprefixallocculmulative}
% 	\caption{Trends in IP prefix allocation}
%  	\label{fig:IP allocations new and gone}
% \end{figure}

Figure \ref{fig:IP allocations new and gone} shows that the number of allocated IP prefixes has increased by almost 40,000 in the past six years. Some allocations have disappeared, but a larger number of new allocations has outpaced the disappearances.

\begin{figure}[htbp]
    \centering
        \includegraphics[width=\columnwidth]{04_3_plus_minus/changes}
    \caption{Changes in IP allocations}
    \label{fig:IP allocations new and gone}
\end{figure}


\subsection{Unaligned allocation}
Unaligned IP allocations are ones whose size is not a power of two. For instance, a block of 1000 addresses could be allocated instead of 1024. The latter could be represented by a single /22 prefix, whereas the former would need a /23, /24, /25, /26, /27, and /29 prefix to express the allocation. Allocations usually have size of a power of two, so a single prefix will suffice. There are not very many unaligned allocations because they are wasteful of BGP table space and have no advantages over aligned ones. Figure \ref{fig:unaligned IP allocations} shows the distribution of such allocations. Most were allocated between 1992 and 1995, although there have been some every year since then. Compared to the size of the entire routing table, these allocations are insignificant, but it is interesting that they exist at all, since RIRs should know better.
\begin{figure}[htbp]
 	\centering
 		\includegraphics[width=0.5\textwidth]{09_alloc_adhoc/adhoc}
	\caption{Unaligned IP allocations per year}
 	\label{fig:unaligned IP allocations}
\end{figure}
\subsection{Allocation by geographical region}

\begin{figure*}[p]
\centering

%%%%%%%%%%%%%%%%%%%%%%%%%%%%%%%%%%%%%%%%%%%%%%%%%%%%%%%%%%%%%%%%%%
%% BGP counts
%%%%%%%%%%%%%%%%%%%%%%%%%%%%%%%%%%%%%%%%%%%%%%%%%%%%%%%%%%%%%%%%%%
\begin{minipage}[b]{0.48\textwidth}
% \begin{figure}[p]
	\centering
		\includegraphics[trim=0 17px 0px 76px,clip=true,width=\columnwidth]{00_maps/ip_count_2003}%
		\hspace{-0.98\columnwidth}%
		\includegraphics[width=1cm]{scale_ip_count}\hspace{-1cm}%
		\hspace{0.98\columnwidth}
	\caption{Geographical distribution of number of allocated IP blocks on \textbf{January 1, 2003}}
	\label{fig:rir prefixes 2003}
% \end{figure}
\end{minipage}%
%
\quad
%
\begin{minipage}[b]{0.48\textwidth}
% \begin{figure}[p]
	\centering
		\includegraphics[trim=0 17px 0px 76px,clip=true,width=\columnwidth]{00_maps/ip_count_2009_2}%
		\hspace{-0.98\columnwidth}%
		\includegraphics[width=1cm]{scale_ip_count}\hspace{-1cm}%
		\hspace{0.98\columnwidth}
	\caption{Geographical distribution of number of allocated IP blocks on \textbf{April 23, 2009}}
	\label{fig:rir prefixes 2009}
% \end{figure}
\end{minipage}

\vspace{0.5cm}

%%%%%%%%%%%%%%%%%%%%%%%%%%%%%%%%%%%%%%%%%%%%%%%%%%%%%%%%%%%%%%%%%%
%% BGP sizes
%%%%%%%%%%%%%%%%%%%%%%%%%%%%%%%%%%%%%%%%%%%%%%%%%%%%%%%%%%%%%%%%%%
\begin{minipage}[b]{0.48\textwidth}
% \begin{figure}[p]
	\centering
		\includegraphics[trim=0 17px 0px 76px,clip=true,width=\columnwidth]{00_maps/ip_size_2003}%
		\hspace{-0.98\columnwidth}%
		\includegraphics[width=1cm]{scale_ip_size}\hspace{-1cm}%
		\hspace{0.98\columnwidth}
	\caption{Geographical distribution of allocated IP space on \textbf{January 1, 2003}}
	\label{fig:rir ip space 2003}
% \end{figure}
\end{minipage}%
%
\quad
%
\begin{minipage}[b]{0.48\textwidth}
% \begin{figure}[p]
	\centering
		\includegraphics[trim=0 17px 0px 76px,clip=true,width=\columnwidth]{00_maps/ip_size_2009_2}%
		\hspace{-0.98\columnwidth}%
		\includegraphics[width=1cm]{scale_ip_size}\hspace{-1cm}%
		\hspace{0.98\columnwidth}
	\caption{Geographical distribution of allocated IP space on \textbf{April 23, 2009}}
	\label{fig:rir ip space 2009}
% \end{figure}
\end{minipage}

\vspace{0.5cm}

%%%%%%%%%%%%%%%%%%%%%%%%%%%%%%%%%%%%%%%%%%%%%%%%%%%%%%%%%%%%%%%%%%
%% Asia region
%%%%%%%%%%%%%%%%%%%%%%%%%%%%%%%%%%%%%%%%%%%%%%%%%%%%%%%%%%%%%%%%%%
\begin{minipage}[b]{0.48\textwidth}
% \begin{figure}[p]
	\centering
		\includegraphics[trim=0 17px 0px 76px,clip=true,width=\columnwidth]{00_maps/ip_asia_2009_prefixes}%
		\hspace{-0.98\columnwidth}%
		\includegraphics[width=1cm]{scale_ip_count}\hspace{-1cm}%
		\hspace{0.98\columnwidth}
	\caption{Geographical distribution of number of allocated IP blocks in Asian region on \textbf{April 23, 2009}}
	\label{fig:rir prefixes asia 2009}
% \end{figure}
\end{minipage}%
%
\quad
%
\begin{minipage}[b]{0.48\textwidth}
% \begin{figure}[p]
	\centering
		\includegraphics[trim=0 17px 0px 76px,clip=true,width=\columnwidth]{00_maps/ip_asia_2009_space}%
		\hspace{-0.98\columnwidth}%
		\includegraphics[width=1cm]{scale_ip_size}\hspace{-1cm}%
		\hspace{0.98\columnwidth}
	\caption{Geographical distribution of allocated IP space in Asian region on \textbf{April 23, 2009}}
	\label{fig:rir ip space asia 2009}
% \end{figure}
\end{minipage}

\end{figure*}

% \clearpage

\begin{table*}[p]
%%%%%%%%%%%%%%%%%%%%%%%%%%%%%%%%%%%%%%%%%%%%%%%%%%%%%%%%%%%%%%%%%%%%%%%%%%%%%%%%
%% TOP announced prefixes
%%%%%%%%%%%%%%%%%%%%%%%%%%%%%%%%%%%%%%%%%%%%%%%%%%%%%%%%%%%%%%%%%%%%%%%%%%%%%%%%
\begin{minipage}[t]{0.48\textwidth}
% \begin{table}[p]
	\begin{center}
	\caption{Top 25 countries with the most number of allocated IP blocks on \textbf{January 1, 2003}}
	\label{tab:top25 rir prefixes 2003}
	\begin{tabular}{|l||l|r|r|}
		\hline
		&      \bf Country		& \bf Prefixes  &  \bf  IP space 		\tabularnewline \hline
1       &       US      		&       31,699  &       1,240,486,995   \tabularnewline
2       &       Canada  		&       5,314   &       61,593,600      \tabularnewline
3       &       Germany 		&       1,642   &       49,413,120      \tabularnewline
4       &       UK      		&       1,573   &       74,358,784      \tabularnewline
5       &       Australia       &       1,351   &       22,956,032      \tabularnewline
6       &       Italy   		&       836     &       14,270,464      \tabularnewline
7       &       Switzerland     &       737     &       10,523,904      \tabularnewline
8       &       Japan   		&       674     &       95,166,320      \tabularnewline
9       &       France  		&       625     &       37,038,080      \tabularnewline
10      &       Netherlands     &       619     &       28,387,328      \tabularnewline
11      &       Sweden  		&       533     &       13,377,024      \tabularnewline
12      &       Russia  		&       501     &       6,259,200       \tabularnewline
13      &       Hong Kong       &       491     &       4,476,416       \tabularnewline
14      &       China   		&       393     &       29,396,736      \tabularnewline
15      &       New Zealand     &       366     &       3,820,288       \tabularnewline
16      &       Finland 		&       348     &       8,085,760       \tabularnewline
17      &       Norway  		&       322     &       8,610,304       \tabularnewline
18      &       Spain   		&       310     &       9,625,344       \tabularnewline
19      &       South Africa    &       275     &       8,163,328       \tabularnewline
20      &       Austria 		&       267     &       5,279,232       \tabularnewline
21      &       Brazil  		&       260     &       10,902,784      \tabularnewline
22      &       Chile   		&       251     &       2,310,656       \tabularnewline
23      &       Singapore       &       250     &       1,933,856       \tabularnewline
24      &       Thailand        &       245     &       1,667,328       \tabularnewline
25      &       India   		&       240     &       2,636,032       \tabularnewline
% 26      &       South Korea     &       197     &       26,208,768      \tabularnewline
% 27      &       Indonesia       &       188     &       1,005,568       \tabularnewline
% 28      &       Taiwan  		&       184     &       11,659,008      \tabularnewline
% 29      &       Poland  		&       174     &       3,982,080       \tabularnewline
% 30      &       Belgium 		&       163     &       4,664,832       \tabularnewline
	\hline
	\end{tabular}
	\end{center}
% \end{table}
\end{minipage}
%
\quad
%
\begin{minipage}[t]{0.48\textwidth}
% \begin{table}[p]
	\begin{center}
	\caption{Top 25 countries with the most number of allocated IP blocks on \textbf{April 23, 2009}}
	\label{tab:top25 rir prefixes 2009}
	\begin{tabular}{|l||l|r|r|r|}
		\hline
		&      \bf Country		& \bf Prefixes  &       \bf IP space 	& \bf Change$^{*}$ 	\tabularnewline \hline 
1       &       US      &       36,881  &       1,473,990,144   &         1.16  \tabularnewline
2       &       Australia       &       6,099   &       37,378,304      &         4.51  \tabularnewline
3       &       Canada  &       5,709   &       75,905,792      &         1.07  \tabularnewline
4       &       Germany &       5,612   &       85,205,400      &         3.42  \tabularnewline
5       &       European Union  &       5,074   &       114,168,224     &        46.98  \tabularnewline
6       &       UK      &       3,732   &       70,756,184      &         2.37  \tabularnewline
7       &       Russia  &       3,148   &       24,607,688      &         6.28  \tabularnewline
8       &       Japan   &       2,068   &       153,285,376     &         3.07  \tabularnewline
9       &       France  &       1,814   &       68,384,704      &         2.90  \tabularnewline
10      &       Ukraine &       1,769   &       5,516,480       &        19.88  \tabularnewline
11      &       Poland  &       1,602   &       13,869,704      &         9.21  \tabularnewline
12      &       China   &       1,566   &       191,643,392     &         3.98  \tabularnewline
13      &       Netherlands     &       1,449   &       21,291,560      &         2.34  \tabularnewline
14      &       Switzerland     &       1,359   &       8,249,320       &         1.84  \tabularnewline
15      &       New Zealand     &       1,217   &       6,116,096       &         3.33  \tabularnewline
16      &       Italy   &       955     &       32,206,272      &         1.14  \tabularnewline
17      &       South Africa    &       886     &       15,057,920      &         3.22  \tabularnewline
18      &       Sweden  &       862     &       18,986,400      &         1.62  \tabularnewline
19      &       Austria &       854     &       7,292,128       &         3.20  \tabularnewline
20      &       Romania &       769     &       8,643,328       &        21.36  \tabularnewline
21      &       Czech Republic  &       706     &       6,059,392       &         5.98  \tabularnewline
22      &       South Korea     &       700     &       72,193,792      &         3.55  \tabularnewline
23      &       Finland &       655     &       8,932,864       &         1.88  \tabularnewline
24      &       Hong Kong       &       651     &       8,208,128       &         1.33  \tabularnewline
25      &       India   &       611     &       18,290,432      &         2.55  \tabularnewline
% 26      &       Spain   &       530     &       21,794,976      &         1.71  \tabularnewline
% 27      &       Denmark &       491     &       9,289,824       &         3.43  \tabularnewline
% 28      &       Indonesia       &       482     &       7,263,488       &         2.56  \tabularnewline
% 29      &       Taiwan  &       422     &       24,680,704      &         2.29  \tabularnewline
% 30      &       Argentina       &       421     &       7,395,072       &         2.75  \tabularnewline
	\hline
	\end{tabular}
	\end{center}

	\small	$^{*}$ -- Relative change in number of allocated IP blocks from January 1, 2003 and April 23, 2009
% \end{table}
\end{minipage}

\vspace{1cm}

%%%%%%%%%%%%%%%%%%%%%%%%%%%%%%%%%%%%%%%%%%%%%%%%%%%%%%%%%%%%%%%%%%%%%%%%%%%%%%%%
%% TOP announced IP space
%%%%%%%%%%%%%%%%%%%%%%%%%%%%%%%%%%%%%%%%%%%%%%%%%%%%%%%%%%%%%%%%%%%%%%%%%%%%%%%%
\begin{minipage}[t]{0.48\textwidth}
% \begin{table}[p]
	\begin{center}
	\caption{Top 25 countries with the most allocated IP space on \textbf{January 1, 2003}}
	\label{tab:top25 rir ip space 2003}
	\begin{tabular}{|l||l|r|r|}
		\hline
		&      \bf Country		& \bf Prefixes  &  \bf IP space 		\tabularnewline \hline 
1       &       US      		&       31,699  &       1,240,486,995   \tabularnewline
2       &       Japan   		&       674     &       95,166,320      \tabularnewline
3       &       UK      		&       1,573   &       74,358,784      \tabularnewline
4       &       Canada  		&       5,314   &       61,593,600      \tabularnewline
5       &       Germany 		&       1,642   &       49,413,120      \tabularnewline
6       &       France  		&       625     &       37,038,080      \tabularnewline
7       &       China   		&       393     &       29,396,736      \tabularnewline
8       &       Netherlands     &       619     &       28,387,328      \tabularnewline
9       &       South Korea     &       197     &       26,208,768      \tabularnewline
10      &       Australia       &       1,351   &       22,956,032      \tabularnewline
11      &       Italy   		&       836     &       14,270,464      \tabularnewline
12      &       Sweden  		&       533     &       13,377,024      \tabularnewline
13      &       Taiwan  		&       184     &       11,659,008      \tabularnewline
14      &       Brazil  		&       260     &       10,902,784      \tabularnewline
15      &       Switzerland     &       737     &       10,523,904      \tabularnewline
16      &       Spain   		&       310     &       9,625,344       \tabularnewline
17      &       Norway  		&       322     &       8,610,304       \tabularnewline
18      &       South Africa    &       275     &       8,163,328       \tabularnewline
19      &       Finland 		&       348     &       8,085,760       \tabularnewline
20      &       Russia  		&       501     &       6,259,200       \tabularnewline
21      &       Mexico  		&       132     &       5,644,288       \tabularnewline
22      &       Austria 		&       267     &       5,279,232       \tabularnewline
23      &       Belgium 		&       163     &       4,664,832       \tabularnewline
24      &       Denmark 		&       143     &       4,634,624       \tabularnewline
25      &       Hong Kong       &       491     &       4,476,416       \tabularnewline
% 26      &       Poland  		&       174     &       3,982,080       \tabularnewline
% 27      &       New Zealand     &       366     &       3,820,288       \tabularnewline
% 28      &       European Union  &       108     &       3,149,824       \tabularnewline
% 29      &       India  			&       240     &       2,636,032       \tabularnewline
% 30      &       Israel 			&       81      &       2,579,712       \tabularnewline
	\hline
	\end{tabular}
	\end{center}
	\ \newline\ \newline
% \end{table}
\end{minipage}
%
\quad
%
\begin{minipage}[t]{0.48\textwidth}
% \begin{table}[p]
	\begin{center}
	\caption{Top 25 countries with the most allocated IP space on \textbf{April 23, 2009}}
	\label{tab:top25 rir ip space 2009}
	\begin{tabular}{|l||l|r|r|r|}
		\hline
		&      \bf Country		& \bf Prefixes  &       \bf IP space 	& \bf Change$^{*}$ 	\tabularnewline \hline 
1       &       US      &       36,881  &       1,473,990,144   &         1.19  \tabularnewline
2       &       China   &       1,566   &       191,643,392     &         6.52  \tabularnewline
3       &       Japan   &       2,068   &       153,285,376     &         1.61  \tabularnewline
4       &       European Union  &       5,074   &       114,168,224     &        36.25  \tabularnewline
5       &       Germany &       5,612   &       85,205,400      &         1.72  \tabularnewline
6       &       Canada  &       5,709   &       75,905,792      &         1.23  \tabularnewline
7       &       South Korea     &       700     &       72,193,792      &         2.75  \tabularnewline
8       &       UK      &       3,732   &       70,756,184      &          .95  \tabularnewline
9       &       France  &       1,814   &       68,384,704      &         1.85  \tabularnewline
10      &       Australia       &       6,099   &       37,378,304      &         1.63  \tabularnewline
11      &       Italy   &       955     &       32,206,272      &         2.26  \tabularnewline
12      &       Brazil  &       267     &       29,754,880      &         2.73  \tabularnewline
13      &       Taiwan  &       422     &       24,680,704      &         2.12  \tabularnewline
14      &       Russia  &       3,148   &       24,607,688      &         3.93  \tabularnewline
15      &       Spain   &       530     &       21,794,976      &         2.26  \tabularnewline
16      &       Mexico  &       156     &       21,503,232      &         3.81  \tabularnewline
17      &       Netherlands     &       1,449   &       21,291,560      &          .75  \tabularnewline
18      &       Sweden  &       862     &       18,986,400      &         1.42  \tabularnewline
19      &       India   &       611     &       18,290,432      &         6.94  \tabularnewline
20      &       South Africa    &       886     &       15,057,920      &         1.84  \tabularnewline
21      &       Poland  &       1,602   &       13,869,704      &         3.48  \tabularnewline
22      &       Turkey  &       283     &       10,515,904      &         4.22  \tabularnewline
23      &       Denmark &       491     &       9,289,824       &         2.00  \tabularnewline
24      &       Finland &       655     &       8,932,864       &         1.10  \tabularnewline
25      &       Romania &       769     &       8,643,328       &        12.95  \tabularnewline
% 26      &       Switzerland     &       1,359   &       8,249,320       &          .78  \tabularnewline
% 27      &       Hong Kong       &       651     &       8,208,128       &         1.83  \tabularnewline
% 28      &       Norway  &       419     &       7,425,584       &          .86  \tabularnewline
% 29      &       Argentina       &       421     &       7,395,072       &         3.87  \tabularnewline
% 30      &       Austria &       854     &       7,292,128       &         1.38  \tabularnewline
	\hline
	\end{tabular}
	\end{center}
	\small	$^{*}$ -- Relative change in allocated IP space from January 1, 2003 and April 23, 2009
% \end{table}
\end{minipage}

\end{table*}

\clearpage


\subsection{Allocation changes by geographical region}
\begin{figure}[htbp]
 	\centering
 		\includegraphics[width=0.5\textwidth]{04_2_plus_minus_countries/plus_minus_2003-01-01}
	\caption{Increase in allocations by region 2003}
 	\label{fig:increase2003}
\end{figure}
\begin{figure}[htbp]
 	\centering
 		\includegraphics[width=0.5\textwidth]{04_2_plus_minus_countries/plus_minus_2004-01-01}
	\caption{Increase in allocations by region 2004}
 	\label{fig:increase2004}
\end{figure}
\begin{figure}[htbp]
 	\centering
 		\includegraphics[width=0.5\textwidth]{04_2_plus_minus_countries/plus_minus_2005-01-01}
	\caption{Increase in allocations by region 2005}
 	\label{fig:increase2005}
\end{figure}
\begin{figure}[htbp]
 	\centering
 		\includegraphics[width=0.5\textwidth]{04_2_plus_minus_countries/plus_minus_2006-01-01}
	\caption{Increase in allocations by region 2006}
 	\label{fig:increase2006}
\end{figure}
\begin{figure}[htbp]
 	\centering
 		\includegraphics[width=0.5\textwidth]{04_2_plus_minus_countries/plus_minus_2007-01-01}
	\caption{Increase in allocations by region 2007}
 	\label{fig:increase2007}
\end{figure}
\begin{figure}[htbp]
 	\centering
 		\includegraphics[width=0.5\textwidth]{04_2_plus_minus_countries/plus_minus_2008-01-01}
	\caption{Increase in allocations by region 2008}
 	\label{fig:increase2008}
\end{figure}
As can be seen in figures \ref{fig:increase2003}-\ref{fig:increase2008}, the increase in allocations each years has no clear trend and does not appear to vary in a consistent way. In each country in can vary significantly from one year to the next. The same countries do not allocated more than others every year. 
