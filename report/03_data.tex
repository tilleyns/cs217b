\section{Data sets and methodology}
\label{sec:data sets}

The aim of our project is to update and extend IP allocation and BGP routing
table measurements performed in 2002--2004
\cite{Meng:2003:An-analysis-of-BGP-routing} \cite{Xu:2003:IPv4-Address}
\cite{Meng:2005:IPv4-address}. Therefore, we have focused on the time period
from January 2003 to April 2009. Due to the enormous amount of allocation and
BGP announcement statistics, we were forced to limit ourselves in our analyses
to a month scale. This limitation has practically no impact on IP allocation
analysis, because the IP allocations are generally static. On the other hand,
due to the dynamic nature of the BGP routing table, collected data does not
represent global routing table contents very precisely. However, the obtained
results are highly consistent with previous analyses
\cite{Meng:2005:IPv4-address} and up-to-date BGP measurements
\cite{::IPv4-Address-Report}. We are also claiming that the obtained results
represent lower bounds of the high level dynamic trends.

% \subsection{Data sets}

IP allocation data was collected from five Regional Internet Registries (RIRs)
\cite{::IANA----Number}, -- ARIN, RIPE NCC, APNIC, LACNIC, AfriNIC, which,
with some exceptions, cover the designated world regions. Due to the fact that
AfriNIC RIR has been officially recognized only on April 8, 2005
\cite{AKPLOGAN:2005:AfriNIC-now-officially}, IP address space assignment for
the African region before 2005 was maintained by all others RIRs. Moreover,
after 2005 some small part of the statistical information was overlapped
between AfriNIC and other RIRs.

BGP data are monitored by two separate data collection projects: the
University of Oregon Route Views project \cite{::Route-Views} and the RIPE NCC
Routing Information Service (RIS) project \cite{::RIS}. These projects
deployed BGP monitors in more than 20 different locations, including the
United States, the European Union, and Japan, which collect data from more
than 600 BGP peers. For our project we collected statistical data from 5
BGP monitors (one in Oregon for RouteViews project and four in Amsterdam,
London, Tokyo, and Moscow for RIS project). All discovered trends are
consistent over all 5 monitors. That's why we, for space conservation
purposes, decided to present BGP analysis based on data collected from
RouteViews project.

% \subsection{Methodology}
%	
% To process high volume of statistical information (over 80,000,000 records
% of BGP data and over 24,000,000 records of IP allocation data), we imported
% all data into a PostgresSQL database server. Data was processed on
% month-by-month basis and is consistent with previous analyses
% \cite{Meng:2005:IPv4-address} (in the period 2003--2004) and up-to-date
% measurements \cite{::IPv4-Address-Report}.
