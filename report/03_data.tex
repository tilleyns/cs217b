\section{Data sets and methodology}
\label{sec:data sets}

The aim of our project is to update and extend IP allocation and BGP routing
table measurements performed in 2002--2004
\cite{Meng:2003:An-analysis-of-BGP-routing} \cite{Xu:2003:IPv4-Address}
\cite{Meng:2005:IPv4-address}. We have therefore focused on the time period
from January 2003 to April 2009. Due to the enormous quantity of allocation and
BGP announcement data, we limited the analysis level on a per-month scale. On the IP allocation analysis, this limitation has no significant impact because IP allocations are generally static. On the other hand,
due to the dynamic nature of the BGP routing table, collected data does not
represent the global routing table contents as precisely. However, the obtained
results are highly consistent with previous analyses
\cite{Meng:2005:IPv4-address} and up-to-date BGP measurements
\cite{::IPv4-Address-Report}. We also assert that the obtained results
represent lower bounds of the high-level dynamic trends.

% \subsection{Data sets}

IP allocation data was collected from five Regional Internet Registries (RIRs)
\cite{::IANA----Number}, -- ARIN, RIPE NCC, APNIC, LACNIC, AfriNIC, which,
with minor exceptions, cover the designated world regions. Because of the fact that
AfriNIC RIR was officially recognized only as of April 8, 2005
\cite{AKPLOGAN:2005:AfriNIC-now-officially}, IP address space assignments for
the African region prior to 2005 were maintained by all other RIRs. Moreover,
after 2005 some small part of the statistical information was overlapping
between AfriNIC and other RIRs.

BGP data are monitored by two separate data collection projects: the
University of Oregon Route Views project \cite{::Route-Views} and the RIPE NCC
Routing Information Service (RIS) project \cite{::RIS}. These projects have
deployed BGP monitors in more than 20 locations, including the
United States, the European Union, and Japan. These monitors collect data from more
than 600 BGP peers. In this project we collected statistical data from 5
BGP monitors (one in Oregon for the Route Views project and four for the RIS project in Amsterdam,
London, Tokyo, and Moscow). All discovered trends are
consistent qualitatively over all 5 monitors. For this reason, it was reasonable to present the findings from just the Route Views project as representative of all our BGP analysis.

% \subsection{Methodology}
%	
% To process high volume of statistical information (over 80,000,000 records
% of BGP data and over 24,000,000 records of IP allocation data), we imported
% all data into a PostgresSQL database server. Data was processed on
% month-by-month basis and is consistent with previous analyses
% \cite{Meng:2005:IPv4-address} (in the period 2003--2004) and up-to-date
% measurements \cite{::IPv4-Address-Report}.
