\section{Related work}
\label{sec:related_work}

Past studies \cite{Meng:2005:IPv4-address} \cite{Xu:2003:IPv4-Address}
\cite{Meng:2003:An-analysis-of-BGP-routing} have characterized growth of the
Border Gateway Protocol routing table in terms of the prevalence of special
announcements to suit traffic engineering purposes, longevity of announcements
appearing and disappearing, and estimated time left until the IPv4 space is
fully allocated. These studies reflect concerns that BGP operates with
functions that are not entirely free of conflict with each other (efficiency
versus policy priorities, i.e. public good versus self serving choices), that
network traffic growth stemming from BGP updates tracking connectivity changes
faces scalability limitations, and that the BGP routing table size contributes
to routing latency. Since the time of those studies between 2003 and 2005, the
BGP routing table has continued its rate of growth. It is helpful to examine
the current state of the BGP routing table and quantify how that high-level
picture has changed from earlier measurements.

Geoff Huston's Potaroo project \cite{::IPv4-Address-Report} present up-to-date
measurements of the BGP routing table growth from 1994. However, it is also
worthwhile analyzing whether table fragmentation or aggregation has changed
over time and how that affects estimates of the routing table size in the
future.

Another aspect, particularly informative and not as well charted in the past,
is to document where routing announcements are originating around the world --
that is, not necessarily a measure of where most new Internet traffic is
occurring but a way to witness the spreading of Internet infrastructure
connectivity around the globe. Coupled with an analysis of how long these
announcements stay in the routing table -- a measure of table stability -- it
is possible to make some projections how the purposes of connecting to the
Internet continue to diversify.

