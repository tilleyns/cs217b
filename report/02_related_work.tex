\section{Related work}
\label{sec:related_work}

Past studies \cite{Meng:2005:IPv4-address} \cite{Xu:2003:IPv4-Address}
\cite{Meng:2003:An-analysis-of-BGP-routing} have characterized growth of the
Border Gateway Protocol routing table in terms of the prevalence of special
announcements to suit traffic engineering purposes. They also measured the
number of appearing and disappearing announcements in the BGP routing table,
the latency between allocation and prefix appearance in BGP announcements, and
the level of unallocated address announcements. Since the time of those
studies between 2003 and 2005, the BGP routing table has continued its growth.
It is helpful to examine the current state of the BGP routing table and
quantify how that high-level picture has changed from earlier measurements.

Geoff Huston's Potaroo project \cite{::IPv4-Address-Report} presents
up-to-date measurements of the BGP routing table growth and IP allocation
dynamics from 1994. However, it is also worthwhile to analyze the impact of
fragmentation and address space duplication on the BGP table growth over time
and how that affects estimates of the routing table size in the future.

Our main contribution is to document where routing announcements are
originating around the world -- not necessarily a measure of where most new
Internet traffic is occurring, but a way to witness the spreading of Internet
infrastructure connectivity around the globe. Coupled with an analysis of how
long these announcements stay in the routing table -- a measure of table
stability -- it is possible to make some projections how the purposes of
connecting to the Internet continue to diversify.

