\documentclass[journal,final]{IEEEtran}%,onecolumn
\usepackage[colorlinks,linkcolor=blue,filecolor=blue,citecolor=red,bookmarksnumbered=true]{hyperref}
\usepackage{cite}
\usepackage{graphicx}
\usepackage[cmex10]{amsmath}
\interdisplaylinepenalty=2500
\usepackage{flushend}

\graphicspath{{../reports/}}

\title{Measuring the Global Routing System}

% \author{
% \IEEEauthorblockN{Alexander~Afanasyev}
% \IEEEauthorblockA{Computer Science Department \\
% University of California, Los Angeles \\
% Email: alexander.afanasyev@ucla.edu}
% \and
% \IEEEauthorblockN{Brent~Longstaff}
% \IEEEauthorblockA{Computer Science Department \\
% University of California, Los Angeles \\
% Email: blongstaff@ucla.edu}
% \and
% \IEEEauthorblockN{Neil~Tilley}
% \IEEEauthorblockA{Computer Science Department \\
% University of California, Los Angeles \\
% Email: tilleyns@ucla.edu}%
% }

\author{
Alexander~Afanasyev, %
%,˜\IEEEmembership{Member,˜IEEE,}
Brent Longstaff, and Neil~Tilley \\
\small \{alexander.afanasyev, blongstaff, tilleyns\}@ucla.edu \\ \ \\
\small Computer Science Department \\
\small University of California, Los Angeles
}

\begin{document}

\maketitle

\begin{abstract} % It is so boring that BGP is currently the most important protocol

BGP is currently the most important protocol for insuring the global
connectivity of the Internet. This places a great deal of responsibility on BGP
and creates a number of challenges for it. A primary concern is the impact of
various currently deployed BGP-based techniques on the scalability of the
global routing table. While these techniques provide ISPs with additional
traffic management services (e.g., balancing, multihoming, etc), they have
expanded the routing table size at a pace that exceeds the allocation rate and
is increasing. In our study we present a two-plane analysis of BGP
announcements for the period 2003--2009. First, we correlate globally announced
prefixes with IP allocation data and show how efficiently ISPs announce their
allocated address space. Second, we correlate BGP announcement data to itself
and show various internal factors contributing to the routing table growth.
Finally, we document in which regions of the world routing announcements have
been originating during the period of this study, and we draw conclusions about
the spread of Internet connectivity globally.

% Second, we note occurrences of IP prefix spaces being announced multiple
% times at increasing levels of detail and show these as one of the various
% factors contributing to the routing table growth.

\end{abstract}

\section{Introduction}

BGP (Border Gateway Protocol) is the critical infrastructure for the Internet
routing. The routing protocol operates at the junction point
where independent networks (ASes, or autonomous systems) exchange network
traffic through proscribed and announced routes of connectivity. Because ASes
are separate networking and economic entities, BGP must operate, while
balancing essentially two purposes, which are not entirely orthogonal to one
another: the limits of delivering efficiently routed network traffic and the
realities of routing policies, which are governed by operating costs, a number
of policy-based and politically-based issues, network locality, multihoming
preferences and, in some select cases, traffic connection capacity. These have
functioned, mostly together and at times as trade-offs to one another, to
increase complexity within BGP and to deliver the present level of routed
Internet traffic -- together with its policy-bound routing inefficiencies --
as is seen at this time.

As originally envisioned, a hierarchical and scalable routing table was to
serve as an efficient and streamlined mechanism. However, not foreseen was the
degree to which the limited number of IPv4 addresses ($2^{32}$) and the
increasing number of allocations to users would lead to fractionalization and
finer segmentation of the IP address space. Fragmentation has effectively
flattened portions of the IP address table, rather than preserved the
hierarchical IP address-based routing. Reasons for numerous "special-case"
announcements include multihoming demands and Internet customers'
implementation of particular traffic engineering to suit any special purposes.
Likewise, individual institutions have grown to need more IP addresses than
originally allocated and received additional address blocks that are
non-adjacent. In either case, the routing table has expanded enormously over
the past ten years, with the table maintaining more entries than a
hierarchical structure would have yielded that worked with strictly
consolidated blocks. Figure~\ref{fig:BGP vs RIR}, for example, that over the
time interval of this current study (2003-2009), the number of table entries
increased from a little of double number of allocated IP blocks to over three
times the number of blocks. To account for this, correspondingly, the Internet
has handled an increasing number of transmitted BGP updates to propagate these
steadily ongoing changes.

\begin{figure}[htbp]
	\centering
		\includegraphics[width=\columnwidth]{01_bgp_ip_size/02_bgp_max_vs_ip}
	\caption{Number of BGP entries compared to number of allocated IP blocks}
	\label{fig:BGP vs RIR}
\end{figure}

There is an interesting aspect to notice with the BGP table growth. While the
routing table has undergone a substantial growth (compared to the number of
new allocated IP spaces), all the IP space that these announcements cover is
still only a fraction of the total measured IP space that has been allocated.
This is illustrated in Figure~\ref{fig:BGP vs RIR space}, which illustrates
that over time, the amount of dormant IP space -- allocated, but not
announced in routing tables -- has ranged from a little over 1/3 to a little
over 1/4 of the total as unused. Despite many fragmented announcements in the
BGP table, the amount of IP space still unused has not dipped below the 25\%
mark yet, of what is available at a given time.

\begin{figure}[htbp]
	\centering
\includegraphics[width=\columnwidth]{01_bgp_ip_size/02_bgp_max_vs_ip_space}
	\caption{Comparing the IP space that has been allocated to the amount of IP space announced in the BGP table}
	\label{fig:BGP vs RIR space}
\end{figure}

In this paper we update and expand on some of the done previously by Meng et
al. \cite{Meng:2005:IPv4-address}. The prior research reported a number of
statistics that serve as a baseline for our project. These include 1) the
number of allocated blocks and 2) the number of announced prefixes. First, we
analyze dynamics of IP address allocation. We share the recent history of
which prefix block sizes are most popular to allocate to ISPs (Internet
Service Providers). We show when IP prefixes have been allocated, as viewed on
a yearly basis, and link this activity to regions of the world where such
allocations have been requested. We then summarize where the most changes have
occurred, which indicate where the busiest Internet areas around the world are
and, more importantly, identify as well the regions globally where there has
been the most rapid development of ISP hosting. Second, we turn from IP prefix
allocation and and examine the correlation to the contents of the BGP routing
table. From the common prefix block sizes to allocate, we share what sized
blocks are more typical to be included in the BGP table. On a related note, we
examine the age, or lifespan, of BGP entries as measured between 2003 and
2009. We present, as before with IP address block allocations, a summary by
geography of which regions of the globe contribute to the BGP table contents
each year. It will be possible to draw some conclusions about a country's
"efficiency" of its IP prefix announcement: a measure of how much IP space
allocated to a country is effectively routed with the fewest announcements in
the BGP routing table. We also discuss a number of factors that contribute to
the marked growth in the routing table.

The paper is organized as follows. Section~\ref{sec:data sets} describes the
data sets and methodology used in our study. Section~\ref{sec:allocations} presents statistics for
IP address allocation and several of the dynamics happening regionally. In
Section~\ref{sec:bgp} we analyze the composition of the BGP table and its changes over
time, as well as the longevity and stability of routing table entries.
Finally, we present related work and conclusions in Sections~\ref{sec:related_work} and \ref{sec:conclusions}.

% The existing Internet fully relies on the BGP protocol~\cite{Rekhter:1995:RFC1771-BGP} to maintain global connectivity. The key element of the BGP is that each participant (autonomous system, AS) announces its IP prefixes which are propagated to the rest of the ASes by means of the protocol. Although the announced set is generally limited by the address blocks allocated to a particular AS by a Regional Internet Registry (RIR), the AS itself decides the granularity of the announcements. In other words, ASes having only one allocated address block can announce multiple prefixes. The two major reasons for this are traffic engineering and multihoming. The popularity of this prefix splitting can be demonstrated by allocation vs announcement statistics. The number of announced prefixes (160K in 2004~\cite{Meng:2005:IPv4-address} growing 300K in 2009~\cite{::BGP-Reports}) is more than two times the number of allocated IPv4 blocks (65K in 2004, 140K in 2009 respectively). These dynamics cast doubt on global routing scalability.
%
% As originally envisioned, a hierarchical and scalable routing table was to serve as an efficient and streamlined mechanism. However, not foreseen was how the limited number ($2^{32}$) of IPv4 addresses and alongside the increasing number of allocations to users would lead to fractionalization and finer segmentation of the IP address table. There exist several solutions which attempt to contain the size of the global routing table.  Accompanying the growth of the IP assignments have been aggregation techniques with the intent to gather a group of prefixes under a more general IP prefix. However, Internet customers with particular traffic engineering and multihoming demands have preferred that these guidelines not be implemented. On the other hand, some customers willing to aggregate are not in a position to do so, due to the inability of RIRs to assign a specific requester a contiguous range of IP addresses. Such is the case, for example, with UCLA which has accumulated eight IPv4 address blocks and is forced to announce eight different prefixes (128.97.0.0/16, 131.179.0.0/16, 149.142.0.0/16, 164.67.0.0/16, 169.232.0.0/16, 192.35.210.0/24, 192.35.225.0/24, 192.154.2.0/24). This granular allocation has been one of the major contributors to the growth of the global routing table. An interesting topic to pursue would be to find the average number of allocated blocks assigned to various ASes.
%
% Meng et al.~\cite{Meng:2005:IPv4-address} reported a number of statistics that will serve as a baseline for our project.  These include 1) the number of allocated blocks and 2) the number of announced prefixes. We propose to conduct more detailed research by country and AS granularity on the ratio and correlation between the number of allocated blocks and the announced prefixes within the BGP routing table. Arguably, any attempt to renumber allocations such that they are less fragmented would reduce both the number of allocations and correspondingly the number of prefixes and size of the BGP table. Our analysis will help to establish an upper bound of the potential BGP routing table reduction if an IPv4 renumbering technique were to be implemented. Additionally, this will justify the necessity of effective IPv6 address assignment and reassignment techniques.

% picture of total # of prefixes
% picture of total # of IP space

% Section~\ref{sec:} provides an introduction to the Border Gateway Protocol.    Section 4 presents statistics for IP address allocation and announcement and the BGP table growth.  Section 5 concerns the trends of fragmentation in the BGP routing table.  Section 6 draws a connection between locality and routing table growth, and it shows in which parts of the globe Internet connectivity has been expanding over the last several years.  Section 7 presents data on the longevity and stability of routing table entries.  Related work is discussed in Section 8, followed by the conclusion in Section 9.   Section 2to enable routing to these IP address blocksthe distribution of IP prefix announcements per region of the worldWe propose to conduct more detailed research by country and AS granularity on the ratio and correlation between the number of allocated blocks and the announced prefixes within the BGP routing table. Arguably, any attempt to renumber allocations such that they are less fragmented would reduce both the number of allocations and correspondingly the number of prefixes and size of the BGP table. Our analysis will help to establish an upper bound of the potential BGP routing table reduction if an IPv4 renumbering technique were to be implemented. Additionally, this will justify the necessity of effective IPv6 address assignment and reassignment techniques. 

\section{Data sets and methodology}
\label{sec:data sets}

The aim of our project is to update and extend IP allocation and BGP routing
table measurements performed in 2002--2004
\cite{Meng:2003:An-analysis-of-BGP-routing} \cite{Xu:2003:IPv4-Address}
\cite{Meng:2005:IPv4-address}. We have therefore focused on the time period
from January 2003 to April 2009. Due to the enormous quantity of allocation and
BGP announcement data, we limited the analysis level on a per-month scale. On the IP allocation analysis, this limitation has no significant impact because IP allocations are generally static. On the other hand,
due to the dynamic nature of the BGP routing table, collected data does not
represent the global routing table contents as precisely. However, the obtained
results are highly consistent with previous analyses
\cite{Meng:2005:IPv4-address} and up-to-date BGP measurements
\cite{::IPv4-Address-Report}. We also assert that the obtained results
represent lower bounds of the high-level dynamic trends.

% \subsection{Data sets}

IP allocation data was collected from five Regional Internet Registries (RIRs)
\cite{::IANA----Number}, -- ARIN, RIPE NCC, APNIC, LACNIC, AfriNIC, which,
with minor exceptions, cover the designated world regions. Because of the fact that
AfriNIC RIR was officially recognized only as of April 8, 2005
\cite{AKPLOGAN:2005:AfriNIC-now-officially}, IP address space assignments for
the African region prior to 2005 were maintained by all other RIRs. Moreover,
after 2005 some small part of the statistical information was overlapping
between AfriNIC and other RIRs.

BGP data are monitored by two separate data collection projects: the
University of Oregon Route Views project \cite{::Route-Views} and the RIPE NCC
Routing Information Service (RIS) project \cite{::RIS}. These projects have
deployed BGP monitors in more than 20 locations, including the
United States, the European Union, and Japan. These monitors collect data from more
than 600 BGP peers. In this project we collected statistical data from 5
BGP monitors (one in Oregon for the Route Views project and four for the RIS project in Amsterdam,
London, Tokyo, and Moscow). All discovered trends are
consistent qualitatively over all 5 monitors. For this reason, it was reasonable to present the findings from just the Route Views project as representative of all our BGP analysis.

% \subsection{Methodology}
%	
% To process high volume of statistical information (over 80,000,000 records
% of BGP data and over 24,000,000 records of IP allocation data), we imported
% all data into a PostgresSQL database server. Data was processed on
% month-by-month basis and is consistent with previous analyses
% \cite{Meng:2005:IPv4-address} (in the period 2003--2004) and up-to-date
% measurements \cite{::IPv4-Address-Report}.


\section{IP address allocation dynamics}
\label{sec:allocations}


\subsection{Allocated IP block sizes}

\subsection{Yearly distribution of IP allocations}

\subsection{Unaligned allocation}

\subsection{Allocation by geographical region}

\begin{figure*}[p]
\centering

%%%%%%%%%%%%%%%%%%%%%%%%%%%%%%%%%%%%%%%%%%%%%%%%%%%%%%%%%%%%%%%%%%
%% BGP counts
%%%%%%%%%%%%%%%%%%%%%%%%%%%%%%%%%%%%%%%%%%%%%%%%%%%%%%%%%%%%%%%%%%
\begin{minipage}[b]{0.48\textwidth}
% \begin{figure}[p]
	\centering
		\includegraphics[trim=0 17px 0px 76px,clip=true,width=\columnwidth]{00_maps/ip_count_2003}%
		\hspace{-0.98\columnwidth}%
		\includegraphics[width=1cm]{scale_ip_count}\hspace{-1cm}%
		\hspace{0.98\columnwidth}
	\caption{Geographical distribution of number of allocated IP blocks on \textbf{January 1, 2003}}
	\label{fig:rir prefixes 2003}
% \end{figure}
\end{minipage}%
%
\quad
%
\begin{minipage}[b]{0.48\textwidth}
% \begin{figure}[p]
	\centering
		\includegraphics[trim=0 17px 0px 76px,clip=true,width=\columnwidth]{00_maps/ip_count_2009_2}%
		\hspace{-0.98\columnwidth}%
		\includegraphics[width=1cm]{scale_ip_count}\hspace{-1cm}%
		\hspace{0.98\columnwidth}
	\caption{Geographical distribution of number of allocated IP blocks on \textbf{April 23, 2009}}
	\label{fig:rir prefixes 2009}
% \end{figure}
\end{minipage}

\vspace{0.5cm}

%%%%%%%%%%%%%%%%%%%%%%%%%%%%%%%%%%%%%%%%%%%%%%%%%%%%%%%%%%%%%%%%%%
%% BGP sizes
%%%%%%%%%%%%%%%%%%%%%%%%%%%%%%%%%%%%%%%%%%%%%%%%%%%%%%%%%%%%%%%%%%
\begin{minipage}[b]{0.48\textwidth}
% \begin{figure}[p]
	\centering
		\includegraphics[trim=0 17px 0px 76px,clip=true,width=\columnwidth]{00_maps/ip_size_2003}%
		\hspace{-0.98\columnwidth}%
		\includegraphics[width=1cm]{scale_ip_size}\hspace{-1cm}%
		\hspace{0.98\columnwidth}
	\caption{Geographical distribution of allocated IP space on \textbf{January 1, 2003}}
	\label{fig:rir ip space 2003}
% \end{figure}
\end{minipage}%
%
\quad
%
\begin{minipage}[b]{0.48\textwidth}
% \begin{figure}[p]
	\centering
		\includegraphics[trim=0 17px 0px 76px,clip=true,width=\columnwidth]{00_maps/ip_size_2009_2}%
		\hspace{-0.98\columnwidth}%
		\includegraphics[width=1cm]{scale_ip_size}\hspace{-1cm}%
		\hspace{0.98\columnwidth}
	\caption{Geographical distribution of allocated IP space on \textbf{April 23, 2009}}
	\label{fig:rir ip space 2009}
% \end{figure}
\end{minipage}

\vspace{0.5cm}

%%%%%%%%%%%%%%%%%%%%%%%%%%%%%%%%%%%%%%%%%%%%%%%%%%%%%%%%%%%%%%%%%%
%% Asia region
%%%%%%%%%%%%%%%%%%%%%%%%%%%%%%%%%%%%%%%%%%%%%%%%%%%%%%%%%%%%%%%%%%
\begin{minipage}[b]{0.48\textwidth}
% \begin{figure}[p]
	\centering
		\includegraphics[trim=0 17px 0px 76px,clip=true,width=\columnwidth]{00_maps/ip_asia_2009_prefixes}%
		\hspace{-0.98\columnwidth}%
		\includegraphics[width=1cm]{scale_ip_count}\hspace{-1cm}%
		\hspace{0.98\columnwidth}
	\caption{Geographical distribution of number of allocated IP blocks in Asian region on \textbf{April 23, 2009}}
	\label{fig:rir prefixes asia 2009}
% \end{figure}
\end{minipage}%
%
\quad
%
\begin{minipage}[b]{0.48\textwidth}
% \begin{figure}[p]
	\centering
		\includegraphics[trim=0 17px 0px 76px,clip=true,width=\columnwidth]{00_maps/ip_asia_2009_space}%
		\hspace{-0.98\columnwidth}%
		\includegraphics[width=1cm]{scale_ip_size}\hspace{-1cm}%
		\hspace{0.98\columnwidth}
	\caption{Geographical distribution of allocated IP space in Asian region on \textbf{April 23, 2009}}
	\label{fig:rir ip space asia 2009}
% \end{figure}
\end{minipage}

\end{figure*}

% \clearpage

\begin{table*}[p]
%%%%%%%%%%%%%%%%%%%%%%%%%%%%%%%%%%%%%%%%%%%%%%%%%%%%%%%%%%%%%%%%%%%%%%%%%%%%%%%%
%% TOP announced prefixes
%%%%%%%%%%%%%%%%%%%%%%%%%%%%%%%%%%%%%%%%%%%%%%%%%%%%%%%%%%%%%%%%%%%%%%%%%%%%%%%%
\begin{minipage}[t]{0.48\textwidth}
% \begin{table}[p]
	\begin{center}
	\caption{Top 25 countries with the most number of allocated IP blocks on \textbf{January 1, 2003}}
	\label{tab:top25 rir prefixes 2003}
	\begin{tabular}{|l||l|r|r|}
		\hline
		&      \bf Country		& \bf Prefixes  &  \bf  IP space 		\tabularnewline \hline
1       &       US      		&       31,699  &       1,240,486,995   \tabularnewline
2       &       Canada  		&       5,314   &       61,593,600      \tabularnewline
3       &       Germany 		&       1,642   &       49,413,120      \tabularnewline
4       &       UK      		&       1,573   &       74,358,784      \tabularnewline
5       &       Australia       &       1,351   &       22,956,032      \tabularnewline
6       &       Italy   		&       836     &       14,270,464      \tabularnewline
7       &       Switzerland     &       737     &       10,523,904      \tabularnewline
8       &       Japan   		&       674     &       95,166,320      \tabularnewline
9       &       France  		&       625     &       37,038,080      \tabularnewline
10      &       Netherlands     &       619     &       28,387,328      \tabularnewline
11      &       Sweden  		&       533     &       13,377,024      \tabularnewline
12      &       Russia  		&       501     &       6,259,200       \tabularnewline
13      &       Hong Kong       &       491     &       4,476,416       \tabularnewline
14      &       China   		&       393     &       29,396,736      \tabularnewline
15      &       New Zealand     &       366     &       3,820,288       \tabularnewline
16      &       Finland 		&       348     &       8,085,760       \tabularnewline
17      &       Norway  		&       322     &       8,610,304       \tabularnewline
18      &       Spain   		&       310     &       9,625,344       \tabularnewline
19      &       South Africa    &       275     &       8,163,328       \tabularnewline
20      &       Austria 		&       267     &       5,279,232       \tabularnewline
21      &       Brazil  		&       260     &       10,902,784      \tabularnewline
22      &       Chile   		&       251     &       2,310,656       \tabularnewline
23      &       Singapore       &       250     &       1,933,856       \tabularnewline
24      &       Thailand        &       245     &       1,667,328       \tabularnewline
25      &       India   		&       240     &       2,636,032       \tabularnewline
% 26      &       South Korea     &       197     &       26,208,768      \tabularnewline
% 27      &       Indonesia       &       188     &       1,005,568       \tabularnewline
% 28      &       Taiwan  		&       184     &       11,659,008      \tabularnewline
% 29      &       Poland  		&       174     &       3,982,080       \tabularnewline
% 30      &       Belgium 		&       163     &       4,664,832       \tabularnewline
	\hline
	\end{tabular}
	\end{center}
% \end{table}
\end{minipage}
%
\quad
%
\begin{minipage}[t]{0.48\textwidth}
% \begin{table}[p]
	\begin{center}
	\caption{Top 25 countries with the most number of allocated IP blocks on \textbf{April 23, 2009}}
	\label{tab:top25 rir prefixes 2009}
	\begin{tabular}{|l||l|r|r|r|}
		\hline
		&      \bf Country		& \bf Prefixes  &       \bf IP space 	& \bf Change$^{*}$ 	\tabularnewline \hline 
1       &       US      &       36,881  &       1,473,990,144   &         1.16  \tabularnewline
2       &       Australia       &       6,099   &       37,378,304      &         4.51  \tabularnewline
3       &       Canada  &       5,709   &       75,905,792      &         1.07  \tabularnewline
4       &       Germany &       5,612   &       85,205,400      &         3.42  \tabularnewline
5       &       European Union  &       5,074   &       114,168,224     &        46.98  \tabularnewline
6       &       UK      &       3,732   &       70,756,184      &         2.37  \tabularnewline
7       &       Russia  &       3,148   &       24,607,688      &         6.28  \tabularnewline
8       &       Japan   &       2,068   &       153,285,376     &         3.07  \tabularnewline
9       &       France  &       1,814   &       68,384,704      &         2.90  \tabularnewline
10      &       Ukraine &       1,769   &       5,516,480       &        19.88  \tabularnewline
11      &       Poland  &       1,602   &       13,869,704      &         9.21  \tabularnewline
12      &       China   &       1,566   &       191,643,392     &         3.98  \tabularnewline
13      &       Netherlands     &       1,449   &       21,291,560      &         2.34  \tabularnewline
14      &       Switzerland     &       1,359   &       8,249,320       &         1.84  \tabularnewline
15      &       New Zealand     &       1,217   &       6,116,096       &         3.33  \tabularnewline
16      &       Italy   &       955     &       32,206,272      &         1.14  \tabularnewline
17      &       South Africa    &       886     &       15,057,920      &         3.22  \tabularnewline
18      &       Sweden  &       862     &       18,986,400      &         1.62  \tabularnewline
19      &       Austria &       854     &       7,292,128       &         3.20  \tabularnewline
20      &       Romania &       769     &       8,643,328       &        21.36  \tabularnewline
21      &       Czech Republic  &       706     &       6,059,392       &         5.98  \tabularnewline
22      &       South Korea     &       700     &       72,193,792      &         3.55  \tabularnewline
23      &       Finland &       655     &       8,932,864       &         1.88  \tabularnewline
24      &       Hong Kong       &       651     &       8,208,128       &         1.33  \tabularnewline
25      &       India   &       611     &       18,290,432      &         2.55  \tabularnewline
% 26      &       Spain   &       530     &       21,794,976      &         1.71  \tabularnewline
% 27      &       Denmark &       491     &       9,289,824       &         3.43  \tabularnewline
% 28      &       Indonesia       &       482     &       7,263,488       &         2.56  \tabularnewline
% 29      &       Taiwan  &       422     &       24,680,704      &         2.29  \tabularnewline
% 30      &       Argentina       &       421     &       7,395,072       &         2.75  \tabularnewline
	\hline
	\end{tabular}
	\end{center}

	\small	$^{*}$ -- Relative change in number of allocated IP blocks from January 1, 2003 and April 23, 2009
% \end{table}
\end{minipage}

\vspace{1cm}

%%%%%%%%%%%%%%%%%%%%%%%%%%%%%%%%%%%%%%%%%%%%%%%%%%%%%%%%%%%%%%%%%%%%%%%%%%%%%%%%
%% TOP announced IP space
%%%%%%%%%%%%%%%%%%%%%%%%%%%%%%%%%%%%%%%%%%%%%%%%%%%%%%%%%%%%%%%%%%%%%%%%%%%%%%%%
\begin{minipage}[t]{0.48\textwidth}
% \begin{table}[p]
	\begin{center}
	\caption{Top 25 countries with the most allocated IP space on \textbf{January 1, 2003}}
	\label{tab:top25 rir ip space 2003}
	\begin{tabular}{|l||l|r|r|}
		\hline
		&      \bf Country		& \bf Prefixes  &  \bf IP space 		\tabularnewline \hline 
1       &       US      		&       31,699  &       1,240,486,995   \tabularnewline
2       &       Japan   		&       674     &       95,166,320      \tabularnewline
3       &       UK      		&       1,573   &       74,358,784      \tabularnewline
4       &       Canada  		&       5,314   &       61,593,600      \tabularnewline
5       &       Germany 		&       1,642   &       49,413,120      \tabularnewline
6       &       France  		&       625     &       37,038,080      \tabularnewline
7       &       China   		&       393     &       29,396,736      \tabularnewline
8       &       Netherlands     &       619     &       28,387,328      \tabularnewline
9       &       South Korea     &       197     &       26,208,768      \tabularnewline
10      &       Australia       &       1,351   &       22,956,032      \tabularnewline
11      &       Italy   		&       836     &       14,270,464      \tabularnewline
12      &       Sweden  		&       533     &       13,377,024      \tabularnewline
13      &       Taiwan  		&       184     &       11,659,008      \tabularnewline
14      &       Brazil  		&       260     &       10,902,784      \tabularnewline
15      &       Switzerland     &       737     &       10,523,904      \tabularnewline
16      &       Spain   		&       310     &       9,625,344       \tabularnewline
17      &       Norway  		&       322     &       8,610,304       \tabularnewline
18      &       South Africa    &       275     &       8,163,328       \tabularnewline
19      &       Finland 		&       348     &       8,085,760       \tabularnewline
20      &       Russia  		&       501     &       6,259,200       \tabularnewline
21      &       Mexico  		&       132     &       5,644,288       \tabularnewline
22      &       Austria 		&       267     &       5,279,232       \tabularnewline
23      &       Belgium 		&       163     &       4,664,832       \tabularnewline
24      &       Denmark 		&       143     &       4,634,624       \tabularnewline
25      &       Hong Kong       &       491     &       4,476,416       \tabularnewline
% 26      &       Poland  		&       174     &       3,982,080       \tabularnewline
% 27      &       New Zealand     &       366     &       3,820,288       \tabularnewline
% 28      &       European Union  &       108     &       3,149,824       \tabularnewline
% 29      &       India  			&       240     &       2,636,032       \tabularnewline
% 30      &       Israel 			&       81      &       2,579,712       \tabularnewline
	\hline
	\end{tabular}
	\end{center}
	\ \newline\ \newline
% \end{table}
\end{minipage}
%
\quad
%
\begin{minipage}[t]{0.48\textwidth}
% \begin{table}[p]
	\begin{center}
	\caption{Top 25 countries with the most allocated IP space on \textbf{April 23, 2009}}
	\label{tab:top25 rir ip space 2009}
	\begin{tabular}{|l||l|r|r|r|}
		\hline
		&      \bf Country		& \bf Prefixes  &       \bf IP space 	& \bf Change$^{*}$ 	\tabularnewline \hline 
1       &       US      &       36,881  &       1,473,990,144   &         1.19  \tabularnewline
2       &       China   &       1,566   &       191,643,392     &         6.52  \tabularnewline
3       &       Japan   &       2,068   &       153,285,376     &         1.61  \tabularnewline
4       &       European Union  &       5,074   &       114,168,224     &        36.25  \tabularnewline
5       &       Germany &       5,612   &       85,205,400      &         1.72  \tabularnewline
6       &       Canada  &       5,709   &       75,905,792      &         1.23  \tabularnewline
7       &       South Korea     &       700     &       72,193,792      &         2.75  \tabularnewline
8       &       UK      &       3,732   &       70,756,184      &          .95  \tabularnewline
9       &       France  &       1,814   &       68,384,704      &         1.85  \tabularnewline
10      &       Australia       &       6,099   &       37,378,304      &         1.63  \tabularnewline
11      &       Italy   &       955     &       32,206,272      &         2.26  \tabularnewline
12      &       Brazil  &       267     &       29,754,880      &         2.73  \tabularnewline
13      &       Taiwan  &       422     &       24,680,704      &         2.12  \tabularnewline
14      &       Russia  &       3,148   &       24,607,688      &         3.93  \tabularnewline
15      &       Spain   &       530     &       21,794,976      &         2.26  \tabularnewline
16      &       Mexico  &       156     &       21,503,232      &         3.81  \tabularnewline
17      &       Netherlands     &       1,449   &       21,291,560      &          .75  \tabularnewline
18      &       Sweden  &       862     &       18,986,400      &         1.42  \tabularnewline
19      &       India   &       611     &       18,290,432      &         6.94  \tabularnewline
20      &       South Africa    &       886     &       15,057,920      &         1.84  \tabularnewline
21      &       Poland  &       1,602   &       13,869,704      &         3.48  \tabularnewline
22      &       Turkey  &       283     &       10,515,904      &         4.22  \tabularnewline
23      &       Denmark &       491     &       9,289,824       &         2.00  \tabularnewline
24      &       Finland &       655     &       8,932,864       &         1.10  \tabularnewline
25      &       Romania &       769     &       8,643,328       &        12.95  \tabularnewline
% 26      &       Switzerland     &       1,359   &       8,249,320       &          .78  \tabularnewline
% 27      &       Hong Kong       &       651     &       8,208,128       &         1.83  \tabularnewline
% 28      &       Norway  &       419     &       7,425,584       &          .86  \tabularnewline
% 29      &       Argentina       &       421     &       7,395,072       &         3.87  \tabularnewline
% 30      &       Austria &       854     &       7,292,128       &         1.38  \tabularnewline
	\hline
	\end{tabular}
	\end{center}
	\small	$^{*}$ -- Relative change in allocated IP space from January 1, 2003 and April 23, 2009
% \end{table}
\end{minipage}

\end{table*}

\clearpage


\subsection{Allocation changes by geographical region}

There is another dynamic occurring while the BGP table size grows steadily, and that concerns table entries appearing and disappearing, obviously with appearances outnumbering disappearances that account for the table growth.  Thus, fourth, we will chart both allocation/deallocations and table prefix appearances/disappearances on a monthly basis.


\section{BGP routing table}
\label{sec:bgp}


\subsection{Announced IP block sizes}

\subsection{Age distribution of BGP entries}
Finally, eighth, we will provide a cumulative distribution function (CDF) of prefix ages, by which to draw conclusions about the stability of the routing table content. These will give information not so much about what is happening, but who is involved and for how long.

\subsection{BGP announcements by geographical region}
One other direction of our BGP routing system study deals with characteristics of the allocations and announcements themselves.  These are in the areas of geography and age.  Some of the new areas we will examine to monitor where allocation activity is happening, and for how long a time on average.  Thus, seventh, we will give a year-on-year comparison of prefix allocations and prefix announcements by major countries, including Russia, Japan, major countries of Asia, and broad regions such as the European Union, North America, and Africa.

\subsection{BGP table growth accelerators}

\subsubsection{IP block fragmentation}
Third, we will build a graph showing both the approximated BGP table size and a calculated table size had there been no fragmenting of allocated blocks.

\subsubsection{Duplicate announcements of IP blocks}
Fifth, we will determine the trends of the percentage of covering prefixes that 1) match, 2) fragment, and 3) aggregate allocations.
Sixth, we will show the dynamics of "Level 1" and "Level 2+" covered prefixes over time. Plus, we will give a comparison of covering prefixes to the number of allocations.  All these measurements concern the overall growth of the BGP routing table.

\section{Related work}
\label{sec:related_work}

Past studies \cite{Meng:2005:IPv4-address} \cite{Xu:2003:IPv4-Address}
\cite{Meng:2003:An-analysis-of-BGP-routing} have characterized growth of the
Border Gateway Protocol routing table in terms of the prevalence of special
announcements to suit traffic engineering purposes, longevity of announcements
appearing and disappearing, and estimated time left until the IPv4 space is
fully allocated. These studies reflect concerns that BGP operates with
functions that are not entirely free of conflict with each other (efficiency
versus policy priorities, i.e. public good versus self serving choices), that
network traffic growth stemming from BGP updates tracking connectivity changes
faces scalability limitations, and that the BGP routing table size contributes
to routing latency. Since the time of those studies between 2003 and 2005, the
BGP routing table has continued its rate of growth. It is helpful to examine
the current state of the BGP routing table and quantify how that high-level
picture has changed from earlier measurements.

Geoff Huston's Potaroo project \cite{::IPv4-Address-Report} present up-to-date
measurements of the BGP routing table growth from 1994. However, it is also
worthwhile analyzing whether table fragmentation or aggregation has changed
over time and how that affects estimates of the routing table size in the
future.

Another aspect, particularly informative and not as well charted in the past,
is to document where routing announcements are originating around the world --
that is, not necessarily a measure of where most new Internet traffic is
occurring but a way to witness the spreading of Internet infrastructure
connectivity around the globe. Coupled with an analysis of how long these
announcements stay in the routing table -- a measure of table stability -- it
is possible to make some projections how the purposes of connecting to the
Internet continue to diversify.



\section{Conclusions}
\label{sec:conclusions}

BGP table has more than doubled in 6 years

The BGP table growth outstrips IP allocation rate

Every industrialized nation is participating in BGP table growth

ISPs prefer to fragment large allocated blocks into smaller chunks, e.g., /24 prefixes account for more than 50\% of routing table

Demand for IP addresses outpaces the rate IPs are returned to RIRs

Multihoming and traffic engineering techniques introduce redundancy in BGP table (58\% in 2009)

BGP table is highly dynamic (only 16\% is static)

% BGP:
% 128.97.0.0/16, 131.179.0.0/16, 149.142.0.0/16, 164.67.0.0/16, 169.232.0.0/16, 192.35.210.0/24, 192.35.225.0/24, 192.154.2.0/24
%
% ARIN:
% 131.179.0.0/16||direct assignment
% 128.97.0.0/16|AS52|direct assignment
% 192.154.2.0/24|AS52|direct assignment
% 192.35.210.0/24|AS52|direct assignment
% 192.35.225.0/24|AS52|direct assignment
% 149.142.0.0/16|AS52|direct assignment
% 164.67.0.0/16|AS52|direct assignment
% 169.232.0.0/16|AS52|reassigned
%
% 2607:F010:0000:0000:0000:0000:0000:0000/32|AS52|direct allocation

\bibliographystyle{IEEEtranS}
\bibliography{../cs217}
\end{document}
