\section{Conclusions}
\label{sec:conclusions}

We have thoroughly analyzed BGP announcement dumps available from University of Oregon RouteViews and RIPE NCC Routing Information Service projects. The general conclusion of our analysis is that average size of the BGP routing table increased more than 2 times in last 6 years. The IP address allocations are also doubled in the same period, but numerically, all new allocated blocks accounts less than 18\% of the actual entries in the BGP routing table. The primary causes of the accelerated BGP routing table growth are allocated IP block fragmentation -- more than 80\% of announced prefixes are parts of allocated IP blocks; and announced space duplication -- more than 54\% of address space is covered at least twice in the global routing table. This highlights an emerged problem of a trend to use the global routing table to serve local interests, e.g., implementing traffic engineering and multihoming techniques. Majority of the globally announced prefixes ($>$50\%) have size /24, which additionally strengthen conclusion that most of the BGP entries serve not to global, but local interests of small customer networks.




Every industrialized nation is participating in BGP table growth

ISPs prefer to fragment large allocated blocks into smaller chunks, e.g., /24 prefixes account for more than 50\% of routing table

Demand for IP addresses outpaces the rate IPs are returned to RIRs

Multihoming and traffic engineering techniques introduce redundancy in BGP table (58\% in 2009)

BGP table is highly dynamic (only 16\% is static)