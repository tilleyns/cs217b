\section{Conclusions}
\label{sec:conclusions}

We have analyzed BGP announcement dumps available from the University of Oregon
Route Views and RIPE NCC Routing Information Service projects. Between 2003 and
2009, a span of 6 years, the average size of the BGP routing table has more
than doubled. Likewise the IP address allocations has also doubled in the same
period. Numerically, however, all the new allocated blocks have added less than
18\% of the total entries in the BGP routing table.

We have identified several primary causes of the accelerated BGP routing table
growth.  They are as follows: First, fragmentation of allocated IP blocks; more
than 80\% of announced prefixes are parts of allocated IP blocks. Second,
announced space duplication; more than 54\% of the address space is covered at
least twice in the global routing table. This duplication highlights an
emerging problematic trend of using the global routing table to serve local
interests, e.g., to implement traffic engineering and multi-provider
connections.

The content analysis of BGP routing table announcements shows that the majority
of the globally announced prefixes ($>$50\%) are of /24 size. This further
strengthens the conclusion that a substantial number of entries in the global
routing table serve local, not global, interests of small customer networks.
The content of the global routing table is highly dynamic. Although there is a
small portion of highly stable entries ($<$15\%), the remainder of the BGP
table content fits an exponential tapering-off distribution for prefix
longevity.

Our examination of the geographical distribution of IP allocation and BGP
announced prefixes shows a depth of penetration of the Internet around the
globe that is wide-ranging. We have observed a number of quasi-exponential
distributions for various measurements, including for the following: the
geographical distribution of the number of allocated prefixes, the numbers of
the corresponding address spaces, the number of announced prefixes, and the
corresponding globally announced address spaces. Moreover these distributions
have not significantly changed in character over the last 6 years.

Another finding is that various countries have differing IP address space
utilization patterns. For example, Japan tends to announce shorter prefixes
(i.e., more addresses covered) than South Korea. Numerically, one announced
prefix belonging to Japan covers 37,200 IP addresses, while in South Korea this
number is only 5,900. Observing this difference can hold relevant implications
for future IPv6 deployment; if additional government regulation can account for
a large part of the difference between the two countries, then it is strongly
worthwhile to consider establishing an analogous and appropriate set of similar
regulations on a global scale.
