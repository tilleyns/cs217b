\section{Conclusions}
\label{sec:conclusions}

We have thoroughly analyzed BGP announcement dumps available from the
University of Oregon RouteViews and RIPE NCC Routing Information Service
projects. The general conclusion of our analysis is that average size of the
BGP routing table more than doubled in last 6 years. The IP address
allocations also doubled in the same period but, numerically, all new
allocated blocks account for less than 18\% of the actual entries in the BGP
routing table. The primary causes of the accelerated BGP routing table growth
are allocated IP block fragmentation (more than 80\% of announced prefixes are
parts of allocated IP blocks) and announced space duplication (more than 54\%
of the address space is covered at least twice in the global routing table).
This highlights an emerging problematic trend of using the global routing
table to serve local interests, e.g., to implement traffic engineering and
multiprovider connections.

The content analysis of the BGP announcements shows that majority of the
globally announced prefixes ($>$50\%) have size /24. This additionally
strengthen the conclusion that most entries in the global routing table serve
not global, but rather local, interests of small customer networks.
Additionally, we conclude that the global routing table is very dynamic.
Although there is a small portion of highly stable entries ($<$15\%), the rest
of the BGP table content shows the exponential distribution of prefix
longevity.

Analysis of the geographical distribution of IP allocation and BGP announced
prefixes shows the varied penetration of Internet on a global scale. The
interesting fact is that the geographical distribution of the number of
allocated prefixes, as well as numbers of corresponding address space, number
of announced prefixes, and corresponding globally announced address space,
displays a quasi-exponential distribution. Moreover, this distribution has not
been changing it's character during last 6 years. Another finding is that
different countries have different IP address space utilization patterns. For
example, Japan tends to announce smaller prefixes (i.e., more addresses
covered) than South Korea. Numerically, one announced prefix belonging to
Japan covers 37,200 IP addresses, while in South Korea this number is only
5,900. If this difference happens because of additional government
regulations, than for future IPv6 deployment, we should consider an
establishment of similar global regulations.
